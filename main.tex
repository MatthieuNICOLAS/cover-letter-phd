%% start of file `template.tex'.
%% Copyright 2006-2013 Xavier Danaux (xdanaux@gmail.com).
%
% This work may be distributed and/or modified under the
% conditions of the LaTeX Project Public License version 1.3c,
% available at http://www.latex-project.org/lppl/.


\documentclass[11pt,a4paper,sans]{moderncv}        % possible options include font size ('10pt', '11pt' and '12pt'), paper size ('a4paper', 'letterpaper', 'a5paper', 'legalpaper', 'executivepaper' and 'landscape') and font family ('sans' and 'roman')

% moderncv themes
\moderncvstyle{classic}                            % style options are 'casual' (default), 'classic', 'oldstyle' and 'banking'
\moderncvcolor{green}                              % color options 'blue' (default), 'orange', 'green', 'red', 'purple', 'grey' and 'black'
%\renewcommand{\familydefault}{\sfdefault}          % to set the default font; use '\sfdefault' for the default sans serif font, '\rmdefault' for the default roman one, or any tex font name
%\nopagenumbers{}                                  % uncomment to suppress automatic page numbering for CVs longer than one page

% character encoding
%\usepackage[utf8]{inputenc}                       % if you are not using xelatex ou lualatex, replace by the encoding you are using
%\usepackage{CJKutf8}                              % if you need to use CJK to typeset your resume in Chinese, Japanese or Korean

% adjust the page margins
\usepackage[scale=0.83]{geometry}
\usepackage{ragged2e}

\definecolor{href}{HTML}{428BCA}
\newcommand{\customboldlink}[1]{\color{href} #1}
\newcommand{\tab}{\quad \quad}

%\setlength{\hintscolumnwidth}{3cm}                % if you want to change the width of the column with the dates
%\setlength{\makecvtitlenamewidth}{10cm}           % for the 'classic' style, if you want to force the width allocated to your name and avoid line breaks. be careful though, the length is normally calculated to avoid any overlap with your personal info; use this at your own typographical risks...

% personal data
\name{Matthieu}{Nicolas}
%\title{Resumé title}                               % optional, remove / comment the line if not wanted
\address{30 Avenue Saint-Sébastien}{54600 Villers-lès-Nancy}{}% optional, remove / comment the line if not wanted; the "postcode city" and and "country" arguments can be omitted or provided empty
%\phone[mobile]{06 75 98 34 40}                   % optional, remove / comment the line if not wanted
%\phone[fixed]{+2~(345)~678~901}                    % optional, remove / comment the line if not wanted
%\phone[fax]{+3~(456)~789~012}                      % optional, remove / comment the line if not wanted
%\email{matthieu.nicolas@inria.fr}                               % optional, remove / comment the line if not wanted
%\homepage{www.johndoe.com}                         % optional, remove / comment the line if not wanted
%\extrainfo{additional information}                 % optional, remove / comment the line if not wanted
%\photo[64pt][0.4pt]{picture}                       % optional, remove / comment the line if not wanted; '64pt' is the height the picture must be resized to, 0.4pt is the thickness of the frame around it (put it to 0pt for no frame) and 'picture' is the name of the picture file
%\quote{Some quote}                                 % optional, remove / comment the line if not wanted

% to show numerical labels in the bibliography (default is to show no labels); only useful if you make citations in your resume
%\makeatletter
%\renewcommand*{\bibliographyitemlabel}{\@biblabel{\arabic{enumiv}}}
%\makeatother
%\renewcommand*{\bibliographyitemlabel}{[\arabic{enumiv}]}% CONSIDER REPLACING THE ABOVE BY THIS

% bibliography with mutiple entries
%\usepackage{multibib}
%\newcites{book,misc}{{Books},{Others}}
%----------------------------------------------------------------------------------
%            content
%----------------------------------------------------------------------------------
\begin{document}
\justify
%-----       letter       ---------------------------------------------------------
% recipient data
\recipient{Equipe COAST}{Centre Inria Nancy\\615 rue du Jardin Botanique\\54600 Villers-lès-Nancy}
\date{À Nancy, le 25 mai 2017}
\opening{
  \textbf{Objet :} candidature à l'offre de thèse portant sur la "(Ré)Identification efficace dans les types de données\\
  \hspace{13mm}    répliquées sans conflit (CRDTs)"
}
\closing{Cordialement,}
%\enclosure[Attached]{curriculum vit\ae{}}          % use an optional argument to use a string other than "Enclosure", or redefine \enclname
\makelettertitle

Monsieur Oster, Monsieur Perrin

% \tab Mon contrat d'ingénieur Recherche et Développement dans le cadre du projet TVPaint arrivant à terme,
% je suis actuellement à la recherche d'un nouvel emploi.
% Souhaitant dorénavant m'orienter vers le domaine de la recherche et de l'enseignement,
% je me permets de vous adresser ma candidature à votre offre de thèse.

\tab Concevant et implémentant depuis plusieurs années des systèmes distribués
reposant sur des mécanismes de réplication de données et de maintien de la cohérence à terme,
le sujet de votre offre de thèse a particulièrement suscité mon intérêt.

%
%   Vous
%   - Reparler du projet
%   - Parler de pourquoi je souhaite rester dans l'équipe
%       - Bonne ambiance
%       - Stimulant intellectuellement
%   - LORIA cadre de travail intéressant
%
\tab Travaillant au LORIA depuis mon stage de fin d'études,
j'ai pu découvrir et me familiariser au fil des mois avec l'environnement de travail proposé par un laboratoire de recherche.
Les sujets de recherche variés, les nombreuses conférences scientifiques organisées
et la diversité des cultures présentes sont autant de raisons
qui font de ce milieu un cadre enrichissant et stimulant dans lequel je souhaite continuer d'évoluer.

%
%   Je - Nous
%   - 3 ans dans l'équipe COAST qui travaille sur la réplication de données
%   - Autonome et capable de s'adapter pour remplir l'ensemble varié des tâches
%   - Compétences dans le domaine des mécanismes de réplication de données
%   - Pu découvrir le monde de la recherche et y a pris goût
%
\tab J'ai rejoint l'équipe COAST au cours de mon stage de fin d'études.
Ce dernier portait sur la réalisation d'un prototype d'éditeur collaboratif temps réel,
\href{https://www.coedit.re}{\customboldlink MUTE}.
Cet éditeur collaboratif implémente \emph{LogootSplit},
un algorithme issu des travaux de l'équipe appartenant à la famille des type de données répliquées sans conflit.
Cette expérience m'a donc permis de découvrir
et de me familiariser avec cette nouvelle approche
de réplication de données et de maintien de la cohérence à terme.

% TODO: Insérer un paragraphe sur la poursuite des travaux autour du MUTE dans le cadre de OpenPaas::NG
% Contribuer au projet OpenPaas:NG
% - Contexte du projet
%   - Réalisation d'une suite d'applications bureautique libre pour entreprises
%   - Vise à proposer une alternative à Google Docs
%   - Rendre cette suite d'applications facilement accessible pour les entreprises (sous forme de services mais aussi facilement déployable)
% - Rôle de l'équipe COAST
%   - Responsable du Work Package intitulé "Infrastructure de collaboration P2P scalable sûre"
%   - En clair, responsable de l'infrastructure P2P (topologie et mise en place du réseau pair à pair)
%     et du mécanisme de réplication de données
% - Mon rôle et ce que ça m'a apporté
%   - Débugger l'implem de LogootSplit
%   - Pu approfondir mes connaissances sur les CRDTs
%     - A regardé d'autres CRDTs pour en utiliser un plus adapté pour la gestion des titres de documents
%   - Développement et mise en place d'un système d'anti-entropie

\tab Par la suite, j'ai continué à travailler sur les CRDTs dans le cadre du projet OpenPaaS::NG.
Ce projet a pour objectif la réalisation d'un réseau social d'entreprise open-source
incorporant une suite d'applications collaboratives pair-à-pair de bureautique.
Le but est ainsi de proposer aux entreprises
une alternative viable et libre à des solutions telles que Google Apps.
L'équipe COAST intervient dans ce projet sur le sujet de
l'échange sécurisé de données dans une architecture pair-à-pair,
mais aussi pour apporter son expertise dans les mécanismes de réplication de données
capable de supporter un grand nombre d'utilisateurs concurrents.
C'est sur cette dernière partie que mes travaux ont portés.
Ainsi, dans le cadre de ce projet, j'ai tout d'abord poursuivi mes travaux
sur l'implémentation de \emph{LogootSplit} en améliorant sa robustesse.
J'ai ensuite étudier une partie de la littérature sur les différents CRDTs existants.
Ceci m'a permis d'acquérir les connaissances nécessaires pour choisir la structure la plus adaptée
en fonction de la nature de la donnée à répliquer et du cas d'utilisation.
Finalement j'ai développé un système d'anti-entropie
permettant de faciliter la convergence de deux répliques~:
en détectant leurs différences, nous sommes capables d'identifier les données
manquantes d'une copie par rapport à l'autre et d'échanger ces dernières.
Ces différents travaux ont été intégrés dans MUTE afin
de pouvoir être validés et présentés aux partenaires du projet.

% TODO: Expliquer que ce sont ces activités de recherche qui m'ont le plus intéressé


% TODO: Retravailler le paragraphe suivant pour insérer ces idées
% - Échange au quotidien avec des doctorants depuis 3 ans
%   - A une bonne idée du job d'un doctorant (tâches, difficultés rencontrées...)
%   - Sais dans quoi je m'engage (bon point ?)
% \tab Finalement, j'ai pu au cours de ces trois dernières années
% échanger au quotidien avec les différents membres de l'équipe COAST.
% Ceci m'a permis de suivre l'évolution de leurs travaux
% mais aussi d'apprendre ce qu'est le métier de doctorant,
% ses tâches ainsi que ses difficultés.
% C'est donc en connaissance de cause que je souhaite mettre en oeuvre
% les connaissances et compétences acquises au cours de ma carrière
% pour contribuer à la conception de nouveaux mécanismes
% permettant de pallier aux limites actuelles des CRDTs.

\tab Finalement, au cours de ma carrière, j'ai pu implémenté des mécanismes issus de la littérature
provenant des deux principales approches utilisées pour la réplication de données dans un système distribué.
Ces expériences m'ont permis au fur et à mesure de prendre connaissance
des spécificités de chacune de ces approches, mais aussi de leurs limites.
Dorénavant, je souhaite poursuivre ma contribution en mettant en oeuvre les connaissances et les compétences acquises
pour concevoir de nouveaux mécanismes permettant de répondre aux problématiques actuelles.

\tab Je reste à votre entière disposition pour un éventuel entretien. Dans l'attente de votre réponse, veuillez agréer, Messieurs, mes salutations distinguées.

%\makeletterclosing

Cordialement,
\begin{flushright}
\textbf{Matthieu Nicolas}
\end{flushright}

\end{document}
%% end of file `template.tex'.
